%----------------------------------------------------------------------------------------
%	PACKAGES AND OTHER DOCUMENT CONFIGURATIONS
%----------------------------------------------------------------------------------------

% \usepackage{XCharter} % Use the Bitstream Charter font
% \usepackage[T1]{fontenc} % Output font encoding for international characters
% \usepackage[default,scale=0.95]{opensans}

% \usepackage[default]{opensans}

\usepackage[none]{hyphenat} % Disable hyphenation
\usepackage{multicol}
\usepackage{hyperref}
\usepackage{cfr-lm}
\usepackage{xifthen}

\usepackage{fontspec}
\setmainfont[%  
  BoldFont      =Open Sans Bold,
  ItalicFont    =Open Sans Italic,
  BoldItalicFont=Open Sans Bold Italic]{Open Sans Regular}
\newfontfamily\light[%
  BoldFont      =Open Sans Semibold,
  ItalicFont    =Open Sans Light Italic,
  BoldItalicFont=Open Sans Semibold Italic]{Open Sans Light}

\usepackage[top=1cm,left=0.7cm,right=0.7cm,bottom=1cm]{geometry} % Modify margins


\usepackage{graphicx} % Required for figures

\usepackage{flowfram} % Required for the multi-column layout

\usepackage{url} % URLs

\usepackage[usenames,dvipsnames]{xcolor} % Required for custom colours

\usepackage{tikz} % Required for the horizontal rule

\usepackage{enumitem} % Required for modifying lists
\setlist{noitemsep,nolistsep} % Remove spacing within and around lists
\setlist[itemize]{leftmargin=*}

\setlength{\columnsep}{\baselineskip} % Set the spacing between columns

\urlstyle{sf}

% Define the left frame (sidebar)
\newflowframe{0.2\textwidth}{\textheight}{0pt}{0pt}[left]
\newlength{\LeftMainSep}
\setlength{\LeftMainSep}{0.2\textwidth}
\addtolength{\LeftMainSep}{1\columnsep}
 
% Small static frame for the vertical line
\newstaticframe{1.5pt}{\textheight}{\LeftMainSep}{0pt}
 
% Content of the static frame with the vertical line
\begin{staticcontents}{1}
\hfill
\tikz{\draw[loosely dotted,color=RoyalBlue,line width=1.5pt,yshift=0](0,0) -- (0,\textheight);}
\hfill\mbox{}
\end{staticcontents}
 
% Define the right frame (main body)
\addtolength{\LeftMainSep}{1.5pt}
\addtolength{\LeftMainSep}{1\columnsep}
\newflowframe{0.7\textwidth}{\textheight}{\LeftMainSep}{0pt}[main01]

\pagestyle{empty} % Disable all page numbering

\setlength{\parindent}{0pt} % Stop paragraph indentation

%----------------------------------------------------------------------------------------
%	NEW COMMANDS
%----------------------------------------------------------------------------------------

\newcommand{\userinformation}[1]{\renewcommand{\userinformation}{#1}} % Define a new command for the CV user's information that goes into the left column

\newcommand{\cvheading}[1]{{\Huge\bfseries\color{RoyalBlue} #1} \par\vspace{.6\baselineskip}} % New command for the CV heading
\newcommand{\cvsubheading}[1]{{\Large\bfseries #1} \bigbreak} % New command for the CV subheading

\newcommand{\Sep}{\vspace{1em}} % New command for the spacing between headings
\newcommand{\SmallSep}{\vspace{0.5em}} % New command for the spacing within headings

\newcommand{\aboutme}[2]{ % New command for the about me section
\textbf{\color{RoyalBlue} #1}~~#2\par\SmallSep
}
	
\newcommand{\CVSection}[1]{ % New command for the headings within sections
\SmallSep
{\Large{\textbf{#1}}}\par\SmallSep % Used for spacing
}



% Parameters
% #1: Company
% #2: Position
% #3: Location
% #4: Date
% #5: Link
% #6: Description
\newcommand{\CVItem}[6]{ % New command for the item descriptions
\CreateTitle{#1}{#2}{#5}{#3} \Date{#4} \par\SmallSep
#6
\InsertSmallSepIfNonEmpty{#6}
}

\newcommand{\Date}[1]{
\hfill{ \small{\textcolor{CadetBlue}{\textit{\sbweight{#1}}}} }
}

\newcommand{\DateLocation}[2]{
\hfill{{\small  \textcolor{CadetBlue}{\sbweight{#2} \hspace*{2pt}} \textit{#1}} }
}


\newcommand{\InsertSmallSepIfNonEmpty}[1]{
    \if\relax\detokenize{#1}\relax
        % Empty
    \else
        % Non-empty
        \par\SmallSep
    \fi
}

% Parameters
% #1: Location
% #2: Description
\newcommand{\InsertLocation}[2]{
    \if\relax\detokenize{#1}\relax
        \par\SmallSep
        #2
    
    \else
        \par
        \hspace*{\fill}{\small\textit{#1}}
        \par\SmallSep
        #2
    \fi
}


% Parameters
% #1: Title
% #2: Position
% #3: Link
% #4: Location
\newcommand{\CreateTitle}[4]{
    \if\relax\detokenize{#2}\relax
        \textbf{\color{RoyalBlue} \link{#3}{#1}}
    \else
        \textbf{\color{RoyalBlue} \link{#3}{#1, \textit{#2}}} 
    \fi
    \if\relax\detokenize{#4}\relax
    \else
        \hspace*{3pt} \textit{#4}
    \fi
}

% Parameters
% #1: Module
% #2: Mark
\newcommand{\Module}[2]{
    \bluebullet #1 \hspace*{2pt}{\small\textit{(#2\%)}}
}

% Same as module, without the %
\newcommand{\ALevel}[2]{ 
    \bluebullet #1 \hspace*{2pt}{\small\textit{(#2)}}
}

\makeatletter
\newcommand*{\link}{\begingroup\@makeother\#\@mylink}
\newcommand*{\@mylink}[2]{\href[pdfnewwindow=true]{#1}{#2}\endgroup} 
\makeatother


\newcommand{\bluebullet}{\textcolor{RoyalBlue}{$\circ$}~~} % New command for the blue bullets
