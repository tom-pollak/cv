%----------------------------------------------------------------------------------------
%	PACKAGES AND OTHER DOCUMENT CONFIGURATIONS
%----------------------------------------------------------------------------------------

\documentclass{structure}

\usepackage[left=0.75in,top=0.6in,right=0.75in,bottom=0.6in]{geometry} % Document margins
\usepackage{xcolor}
\usepackage{fontspec}
\setmainfont[]{Roboto}
% \setmainfont[]{Arial}

\name{Tom Pollak}

\begin{document}

\parbox{0.5\textwidth}{
    {\namesize\textbf{Tom Pollak}} \\[6pt]
    Bristol, UK\\
    \href{mailto:tompollak1000@gmail.com}{tompollak1000@gmail.com}
}
\hfill
\parbox{0.5\textwidth}{
    \vspace*{10pt}

    \begin{flushright}

        \href{https://github.com/tom-pollak}{github.com/tom-pollak} \\
        \href{https://tom-pollak.github.io}{tom-pollak.github.io} \\
        (+44) 77400 54268
    \end{flushright}

}

\smallskip
\hrule
\smallskip

%----------------------------------------------------------------------------------------
%	EXPERIENCE
%----------------------------------------------------------------------------------------

\begin{rSection}{Experience}

    \begin{rSubsection}{Cisco Meraki}{June 2023 -- Present}{ML Software Engineer -- Camera Intelligence Team}{London / Remote, UK}{}{}
        \item Independently researched, designed, \& implemented a novel unified search experience for the Meraki MV cameras using a custom embedding model, that runs and searches real-time with limited resources on camera.

        \item Currently technical lead of a team of 6. I manage camera firmware, model training, inference, system design. Product is set to be the "market differentiator" for the entire MV Camera platform and demoed at Cisco Live 2025.

        \item Developed production-ready 10K+ LOC C++ library and with full testing suite \& benchmarks in GTest.

        \item Built and curated dataset of >200K objects and synthetic NLP queries, enabling classification and text-based search capabilities on camera.

        \item Lead a team at the Cisco San Francisco Hackweek 2023 for preliminary work on the project.

        \item Originally hired as a summer intern in 2022, rejoined full-time in 2023. \href{https://gist.githubusercontent.com/tom-pollak/1a2e8c1fc61ba269e25c73c02c78007c/raw/45c8cbceda8cd745d6d00cb16a09979778df663b/gistfile1.txt}{{\color{blue}Intern Reference}}.
    \end{rSubsection}

\end{rSection}

%----------------------------------------------------------------------------------------
%	EDUCATION
%----------------------------------------------------------------------------------------

\begin{rSection}{Education}

    \begin{rSubsectionNoList}{University of York}{June 2023}{
            BEng. Computer Science -- First Class with Honours
        }{}{}
    \end{rSubsectionNoList}

    \begin{rSubsectionNoList}{Lady Manners School}{2020}{
            A-Level --  Further Maths (A), Maths (A), Computer Science (A), Physics (A)
        }{}{}
    \end{rSubsectionNoList}

\end{rSection}


%----------------------------------------------------------------------------------------
%	PROJECTS
%----------------------------------------------------------------------------------------

\begin{rSection}{Projects}

    \begin{rSubsection}{Interpretability Research}{August 2024 -- Present}{}{}{}{}

        \item Current project: Using \href{https://transformer-circuits.pub/2024/crosscoders/index.html}{Anthropic's Crosscoders} to understand how a LoRA changes a model's behaviour.

        \item Contributed to the \href{https://github.com/jbloomAus/SAELens}{SAELens} library, (\href{https://github.com/jbloomAus/SAELens/pull/321}{\#321}, \href{https://github.com/jbloomAus/SAELens/pull/367}{\#367}) to improve caching activations using Huggingface datasets.

        \item First project: Investigated GPTs trained on 2D grid puzzles similar to ARC-AGI tasks. Trained an SAE, found features related to an individual puzzle task. \href{https://github.com/tom-pollak/interpretability-culture}{GitHub} \hspace{1mm} | \hspace{1mm} \href{https://docs.google.com/document/d/1km2m3oWZMDrekV9_mYHft5pX9PjrM4imKGEdK9vVMr8/edit?usp=sharing}{Report}.

    \end{rSubsection}

    \begin{rSubsection}{nanoViT}{November 2024}{}{}{https://github.com/tom-pollak/nanovit}{}
        \item Minimal ViT implementation \& training from scratch in PyTorch -- {\color{blue}\href{https://colab.research.google.com/github/tom-pollak/nanoViT/blob/main/tutorials/vit_from_scratch.ipynb}{Created a tutorial for building a ViT from scratch.}}

        \item Single GPU training on CIFAR-10, working on multi-GPU replication of \href{https://arxiv.org/abs/2205.01580}{Better plain ViT baselines for ImageNet-1k}.

    \end{rSubsection}

    \begin{rSubsection}{Claudette Pydantic}{July 2024}{}{}{https://github.com/tom-pollak/claudette-pydantic}{}
        \item Adds Pydantic structured outputs through tool use for the \href{https://github.com/AnswerDotAI/claudette}{Claudette} library -- {\color{blue}\href{https://nbviewer.org/github/tom-pollak/claudette-pydantic/blob/main/nbs/examples/pet_store.ipynb}{Example}}.
    \end{rSubsection}

    \begin{rSubsection}{NLP Image Retrieval with CLIP}{September 2022 -- June 2023}{}{}{https://tom-pollak.github.io/clip-index}{https://tom-pollak.github.io/clip-index/assets/Enhancing\%20Image\%20Retrieval\%20in\%20Natural\%20Language\%20Processing\%20Applications.pdf}
        \item University dissertation project -- graded 80\%. Built efficient NLP image search using CLIP and Faiss.
    \end{rSubsection}

    \begin{rSubsection}{Automated Horse Betting Software}{December 2020 -- July 2021}{}{}{https://github.com/tom-pollak/each-way-matcher}{}
        \item Discovers undervalued odds in "each-way" betting, uses an adapted 3-way Kelly Criterion strategy and expected logarithmic growth rate for ranking bets and stake.
        \item Calculates conditional probability of a horse finishing in a given place using the win odds of all horses.
    \end{rSubsection}

\end{rSection}

% %----------------------------------------------------------------------------------------
% %	Courses
% %----------------------------------------------------------------------------------------

% \begin{rSection}{Courses}

%     \begin{rSubsectionNoList}{Mastering LLMs}{June 2024}{}{}{}{}
%     \end{rSubsectionNoList}

%     \begin{rSubsectionNoList}{SANS Institute -- FOR500 Windows Forensic Analysis}{August 2020}{}{}{}{}
%     \end{rSubsectionNoList}

% \end{rSection}

%----------------------------------------------------------------------------------------
%	SKILLS
%----------------------------------------------------------------------------------------

\begin{rSection}{Skills}

    \begin{tabular}{ @{} >{\bfseries}l @{\hspace{6ex}} l }
        Languages & Python, C++, Cuda C.                                                                              \\
        ML        & PyTorch, Numpy, Pandas, einops, Faiss, Huggingface, scikit-learn, Pydantic, Axolotl.              \\
        Tools     & Linux, VSCode, Neovim, Docker, Git, SQLite, {\fontfamily{lmr}\selectfont\LaTeX}.                  \\
    \end{tabular}

\end{rSection}

\end{document}
