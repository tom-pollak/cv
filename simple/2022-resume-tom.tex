%----------------------------------------------------------------------------------------
%	PACKAGES AND OTHER DOCUMENT CONFIGURATIONS
%----------------------------------------------------------------------------------------

\documentclass{structure} 

\usepackage[left=0.75in,top=0.6in,right=0.75in,bottom=0.6in]{geometry} % Document margins
\usepackage{xcolor}
\usepackage{fontspec}
\setmainfont[]{Roboto}
% \setmainfont[]{Arial}

\name{Tom Pollak} 

\begin{document}


\parbox{0.5\textwidth}{ 
{\namesize\bf{Tom Pollak}} \\[6pt]
York, UK\\ 
\href{mailto:tompollak1000@gmail.com}{tompollak1000@gmail.com}
}
\hfill 
\parbox{0.5\textwidth}{ 
\vspace*{10pt}

\begin{flushright}

\href{https://github.com/tom-pollak}{github.com/tom-pollak} \\
\href{https://tompollak.me}{tompollak.me} \\
(+44) 77400 054268
\end{flushright}

}

\smallskip
\hrule
\smallskip
%----------------------------------------------------------------------------------------
%	EDUCATION SECTION
%----------------------------------------------------------------------------------------

\begin{rSection}{Education}

\begin{rSubsection}{University of York}{2023}{BEng. Computer Science}{}{}
    \item Weighted average: 77\% 
\end{rSubsection}

\begin{rSubsection}{Lady Manners School}{2020}{A-Level}{}{}
    \item Further Maths (A), Maths (A), Computer Science (A), Physics (A).
\end{rSubsection}
\end{rSection}

%----------------------------------------------------------------------------------------
%	WORK EXPERIENCE SECTION
%----------------------------------------------------------------------------------------

\begin{rSection}{Experience}

\begin{rSubsection}{Cisco Meraki}{June - August 2022}{Software Engineer Intern - Camera Intelligence}{London, UK}{}
    \item Worked on the attribute search project: designed and implemented an 
    NLP based image search model which allows users to use a text box to 
    search for any object in a video feed, e.g. "A man wearing a blue hat riding a bike"
    \item Used \emph{\href{https://openai.com/blog/clip/}{OpenAI's CLIP}} derivative
    \emph{\href{https://github.com/DRSY/MoTIS}{MoTIS}} to encode images into vectors and
    \emph{\href{https://github.com/spotify/annoy}{Spotify's Annoy library}} to index the images.
    \item Extended the current on-camera motion detection pipeline to feed image "blobs" into the new CLIP model.
    \item Implemented in C++, using PyTorch and NCNN machine learning frameworks.
    \item Using my approach, a busy 10 minute video could be queried in 0.2 seconds --
    faster than most non-NLP models.
    \item \href{https://gist.githubusercontent.com/tom-pollak/1a2e8c1fc61ba269e25c73c02c78007c/raw/45c8cbceda8cd745d6d00cb16a09979778df663b/gistfile1.txt}{\color{blue}{Reference}}
\end{rSubsection}
\end{rSection}

%----------------------------------------------------------------------------------------
%	PROJECTS SECTION
%----------------------------------------------------------------------------------------

\begin{rSection}{Projects}

\begin{rSubsection}{Automated Horse Betting Software}{December 2020 - July 2021}{}{}{https://github.com/tom-pollak/each-way-matcher}
    \item Discovers undervalued horses by the bookmaker in each-way betting.
    \item Exploits the idea that the bookmaker calculates the odds of a horse
    "placing" using only the win odds of the horse, without data from the
    other horses in the race.
    \item Uses an adapted Kelly Criterion strategy with Expected Growth to calculate
    the optimal stake.
    \item Uses Python, Pandas and Selenium to scrape the horse races, interacts with
    Betfair API to place bets.
    \item Runs headless on a Raspberry Pi as a scheduled cron job every day.
\end{rSubsection}

\begin{rSubsection}{Poker Web Application}{April 2019 - July 2020}{}{}{https://github.com/tom-pollak/web-poker}
    \item Free live poker web app using Python, Django and a Postgres database.
    \item Users can create accounts and tables, play poker, and chat with other players.
    \item Implements web-sockets using Django Channels and Redis for real-time communication with the users.
    \item Deployed with Docker and Heroku.
\end{rSubsection}

\begin{rSubsection}{SANS Institute}{August 2020}{FOR500 Windows Forensic Analysis}{}{https://www.sans.org/cyber-security-courses/windows-forensic-analysis}
    \item Sponsored through my success in the Cyber Discovery programme.
\end{rSubsection}

\begin{rSubsection}{Cyber Discovery}{September 2018 - July 2019}{}{}{}
    \item Independently completed the Cyber Discovery programme, run by HM government.
    \item Selected as one of the top 500 (of 28,000) students to attend the Cyber Discovery Elite event in London.
\end{rSubsection}

\end{rSection}

%----------------------------------------------------------------------------------------
%	TECHNICAL STRENGTHS SECTION
%----------------------------------------------------------------------------------------

\begin{rSection}{Skills}

\begin{tabular}{ @{} >{\bfseries}l @{\hspace{6ex}} l }
Languages & Python, C++, Rust, Haskell, SQL, HTML. \\
Tools & Linux, NeoVim, Git, VSCode, JetBrains Suite, RegEx, SQLite, {\fontfamily{lmr}\selectfont\LaTeX}. \\
Technologies & PyTorch, Django, Numpy, Pandas, Selenium, GitHub, Docker. \\
\end{tabular}

\end{rSection}

\end{document}
