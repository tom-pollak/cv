%----------------------------------------------------------------------------------------
%	PACKAGES AND OTHER DOCUMENT CONFIGURATIONS
%----------------------------------------------------------------------------------------

\documentclass{structure} 

\usepackage[left=0.75in,top=0.6in,right=0.75in,bottom=0.6in]{geometry} % Document margins
\usepackage{xcolor}
\usepackage{fontspec}
\setmainfont[]{Roboto}
% \setmainfont[]{Arial}

\name{Tom Pollak} 

\begin{document}


\parbox{0.5\textwidth}{ 
{\namesize\bf{Tom Pollak}} \\[6pt]
York, UK\\ 
\href{mailto:tompollak1000@gmail.com}{tompollak1000@gmail.com}
}
\hfill 
\parbox{0.5\textwidth}{ 
\vspace*{10pt}

\begin{flushright}

\href{https://github.com/tom-pollak}{github.com/tom-pollak} \\
\href{https://tompollak.me}{tompollak.me} \\
(+44) 77400 054268
\end{flushright}

}

\smallskip
\hrule
\smallskip
%----------------------------------------------------------------------------------------
%	EDUCATION SECTION
%----------------------------------------------------------------------------------------

\begin{rSection}{Education}

\begin{rSubsection}{University of York}{2023}{BEng. Computer Science}{}{}
    \item Weighted average: 77\% 
\end{rSubsection}

\begin{rSubsection}{Lady Manners School}{2020}{A-Level}{}{}
    \item Further Maths (A), Maths (A), Computer Science (A), Physics (A).
\end{rSubsection}
\end{rSection}

%----------------------------------------------------------------------------------------
%	WORK EXPERIENCE SECTION
%----------------------------------------------------------------------------------------

\begin{rSection}{Experience}

\begin{rSubsection}{Cisco Meraki}{June - August 2022}{Software Engineer Intern - Camera Intelligence}{London, UK}{}
    \item Worked with the camera team on the attribute search project, designed and implemented an NLP based image search model
    which would allow users to use a text box to search for any object in a video feed.
    \item Uses \href{https://openai.com/blog/clip/}{OpenAI's CLIP model} derivative \href{https://github.com/DRSY/MoTIS}{MoTIS}
    to encode images and \href{https://github.com/DRSY/MoTIS}{Spotify's Annoy} library to index the images. I used the current motion
    detector model to feed images into the CLIP model.
    \item Implemented pipeline in C++, used Pytorch and NCNN model frameworks.
    \item Using my approach, 10 minute video could be queried in 0.2 seconds -- faster than most non-NLP based models.
    \item \href{https://gist.githubusercontent.com/tom-pollak/1a2e8c1fc61ba269e25c73c02c78007c/raw/45c8cbceda8cd745d6d00cb16a09979778df663b/gistfile1.txt}{\color{blue}{Reference}}
\end{rSubsection}
\end{rSection}

%----------------------------------------------------------------------------------------
%	PROJECTS SECTION
%----------------------------------------------------------------------------------------

\begin{rSection}{Projects}

\begin{rSubsection}{Automated Horse Betting Program}{December 2020 - July 2021}{}{}{https://github.com/tom-pollak/each-way-matcher}
    \item Discovers undervalued horses by the bookmaker in each-way betting.
    \begin{list}{$\sbullet[0.6]$}{\leftmargin=1em} 
    \itemsep -0.5em \vspace{-0.5em} 
        \item Exploits the idea that the bookmaker calculates the odds of a horse placing using only the win odds of the horse, without using data from the other horses in the race.
    \end{list}
    \item Uses an adapted Kelly Criterion strategy with Expected Growth to calculate the optimal stake.
    \item Uses Python, Pandas and Selenium to scrape the horse races, interacts with Betfair API to place bets.
    \item Runs headless on a Raspberry Pi as a scheduled cron job every day.
\end{rSubsection}

\begin{rSubsection}{Poker Web Application}{April 2019 - July 2020}{}{}{https://github.com/tom-pollak/web-poker}
    \item Free live poker web app using Python and Django and a Postgres database.
    \item Users can create accounts and tables, play poker, and chat with other players.
    \item Implements websockets using Django Channels and Redis for real-time communication with the users.
    \item Deployed with Docker and Heroku.
\end{rSubsection}

\begin{rSubsection}{SANS Institute}{August 2020}{FOR500 Windows Forensic Analysis}{}{https://www.sans.org/cyber-security-courses/windows-forensic-analysis}
    \item Sponsored through my success in the Cyber Discovery programme.
\end{rSubsection}

\begin{rSubsection}{Cyber Discovery}{September 2018 - July 2019}{}{}{}
    \item Independently completed the Cyber Discovery programme, run by HM government.
    \item Selected as one of the top 500 (of 28,000) students to attend the Cyber Discovery Elite event in London.
\end{rSubsection}

\end{rSection}

%----------------------------------------------------------------------------------------
%	TECHNICAL STRENGTHS SECTION
%----------------------------------------------------------------------------------------

\begin{rSection}{Skills}

\begin{tabular}{ @{} >{\bfseries}l @{\hspace{6ex}} l }
Languages & Python, C++, Rust, Haskell, Java, SQL, HTML. \\
Tools & Linux, Vim, Git, VSCode, JetBrains Suite, RegEx, SQLite, {\fontfamily{lmr}\selectfont\LaTeX}. \\
Technologies & Pytorch, Django, Numpy, Pandas, Selenium, LibGDX, GitHub, Docker. \\
\end{tabular}

\end{rSection}

\end{document}
