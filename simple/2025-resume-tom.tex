%----------------------------------------------------------------------------------------
%	PACKAGES AND OTHER DOCUMENT CONFIGURATIONS
%----------------------------------------------------------------------------------------

\documentclass{structure}

\usepackage[left=0.75in,top=0.6in,right=0.75in,bottom=0.6in]{geometry} % Document margins
\usepackage{xcolor}
\usepackage{fontspec}
\setmainfont[]{Roboto}

\begin{document}

\parbox{0.5\textwidth}{
    {\namesize\textbf{Tom Pollak}} \\[6pt]
    Bristol, UK\\
    \href{mailto:tompollak1000@gmail.com}{tompollak1000@gmail.com}
}
\hfill
\parbox{0.5\textwidth}{
    \vspace*{10pt}

    \begin{flushright}

        \href{https://github.com/tom-pollak}{github.com/tom-pollak} \\
        \href{https://tom-pollak.github.io}{tom-pollak.github.io} \\
        (+44) 77400 54268
    \end{flushright}

}

\smallskip
\hrule
\smallskip

%----------------------------------------------------------------------------------------
%	EXPERIENCE
%----------------------------------------------------------------------------------------

\begin{rSection}{Experience}
    \begin{rSubsection}{Graphcore}{April 2025 -- Present}{Machine Learning Engineer -- Applied AI}{Bristol, UK}{}{}
        \item At Graphcore we build next-gen accelerators, and develop the ecosystem for more heterogeneous compute.

        \item Writing Triton kernels targeting our hardware. Authored fused performant kernels for MoE, Flash Attention, RoPe. Performance profiles help inform Triton compiler and PyTorch teams.

        \item Presented workshop paper on Bayesian inference: {\color{blue}\href{https://openreview.net/pdf?id=JPXNOiOiod}{Variational Entropy Search is Just 1D Regression}} at NeurIPS.

        \item Helping develop pre-training infrastructure, working on load-balancing large MoE models.

        \item Contributed to PyTorch: PyTorch PP deadlock bug when using Gloo ({\color{blue}\href{https://github.com/pytorch/pytorch/pull/152938}{\#152938}}), fix SDPA MATH backend reference implementenation: {\color{blue}\href{https://github.com/pytorch/pytorch/pull/163508}{(\#163508)}}.

    \end{rSubsection}

    \begin{rSubsection}{Cisco Meraki}{June 2023 -- April 2025}{Machine Learning Engineer -- Camera Intelligence Team}{London / Remote, UK}{}{}
        \item At Cisco, I focused on building cross-camera tracking over the 2 years, which was just {\color{blue}\href{https://www.linkedin.com/posts/cisco-enterprise-networking_announcing-the-beta-launch-of-cross-camera-activity-7401681096154980354-ToxM/}{released in Beta.}}

        \item Technical lead of a team of 6 engineers personally managing firmware, model training, inference optimization and architecture; product presented at Cisco Live 2025.

        \item Designed and implemented firmware for high-performance C++ inference engine and scalable distributed k-NN search system across mesh network of cameras (10K+ LOC).

        \item This enabled real-time search \& retrieval that scales to thousands of devices per network with no hit to the backend.

        \item Created multimodal dataset (>200K objects with a mix of synthetic and human labelled annotations) and fine-tuned CLIP-based models for zero-shot object retrieval.
    \end{rSubsection}

\end{rSection}


\begin{rSubsectionNoList}{University of York}{June 2023}{
        BEng. Computer Science -- First Class with Honours
    }{}{}
\end{rSubsectionNoList}


%----------------------------------------------------------------------------------------
%	PROJECTS
%----------------------------------------------------------------------------------------

\begin{rSection}{Projects}
    \begin{rSubsection}{NVFP4 Triton Kernels}{March 2025}{}{}{https://github.com/tom-pollak/xverify}{}
        \item test
    \end{rSubsection}

    \begin{rSubsection}{Minimal reproduction of Diffusion LLMs}{March 2025}{}{}{https://github.com/tom-pollak/dllm/}{}
        \item foo
    \end{rSubsection}

    \begin{rSubsection}{Blender Copilot}{March 2025}{}{}{https://github.com/tom-pollak/blender-copilot}{}
        \item test
    \end{rSubsection}


    \begin{rSubsection}{Structured Generation for LLMs with RLVR}{March 2025}{}{}{https://github.com/tom-pollak/xverify}{}
        \item Developing a library for structured generation and tool use using automatically generated GBNF grammars and Pydantic schema validation for RLVR.
    \end{rSubsection}

    \begin{rSubsection}{Interpretability Research}{August 2024 -- January 2025}{}{}{https://github.com/tom-pollak/interpretability-culture}{}

        \item Investigating features in neural networks trained on ARC-AGI-style 2D grid puzzles

        \item Trained sparse autoencoders (SAEs), discovering task-specific feature in the models, ablating would degrade performance in a specific task.

        \item Applying \href{https://transformer-circuits.pub/2024/crosscoders/index.html}{Anthropic's Crosscoders} to understand how a model changes throughout training.

        \item Contributed to the \href{https://github.com/jbloomAus/SAELens}{SAELens} library: Optimized activation caching with HuggingFace datasets. (PRs \href{https://github.com/jbloomAus/SAELens/pull/321}{\#321}, \href{https://github.com/jbloomAus/SAELens/pull/367}{\#367})

    \end{rSubsection}

    \begin{rSubsection}{Claudette Pydantic}{July 2024}{}{}{https://github.com/tom-pollak/claudette-pydantic}{}
        \item Extended the \href{https://github.com/AnswerDotAI/claudette}{Claudette} library with structured outputs via tool use -- {\color{blue}\href{https://nbviewer.org/github/tom-pollak/claudette-pydantic/blob/main/nbs/examples/pet_store.ipynb}{Example}}.
    \end{rSubsection}

    \begin{rSubsectionNoList}{NLP Image Retrieval with CLIP \& Faiss}{September 2022 -- June 2023}{}{}{https://tom-pollak.github.io/clip-index}{}
    \end{rSubsectionNoList}

    \begin{rSubsection}{Algorithmic Trading System -- Horse Racing}{December 2020 -- July 2021}{}{}{https://github.com/tom-pollak/each-way-matcher}{}
        \item Developed statistical arbitrage system identifying mispriced "each-way" bets.

        \item 3-way Kelly Criterion strategy for optimal stake sizing based on calculated conditional place probabilities.

        \item Successful with high ROI, but low volume and I got banned from profitable bookmakers.
    \end{rSubsection}

\end{rSection}

%----------------------------------------------------------------------------------------
%	SKILLS
%----------------------------------------------------------------------------------------

\begin{rSection}{Skills}

    \begin{tabular}{ @{} >{\bfseries}l @{\hspace{6ex}} l }
        Languages & Python, C++.                                           \\
        ML        & PyTorch, Triton, TorchTitan, Faiss, Slurm, Kubernetes. \\
    \end{tabular}

\end{rSection}

\end{document}
